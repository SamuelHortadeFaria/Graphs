\documentclass[a4paper,12pt]{article}
\usepackage[utf8]{inputenc}  % Codificação UTF-8
\usepackage[T1]{fontenc}     % Suporte a caracteres especiais
\usepackage[brazil]{babel}   % Idioma português
\usepackage{geometry}        % Ajuste de margens
\geometry{a4paper, top=3cm, bottom=2cm, left=3cm, right=2cm}
\usepackage{setspace}        % Controle de espaçamento
\setstretch{1.5}             % Espaçamento 1,5
\usepackage{titlesec}        % Personalização de títulos
\usepackage{graphicx}        % Inclusão de imagens
\usepackage{hyperref}        % Hiperlinks
\hypersetup{
    colorlinks=true,
    linkcolor=black,
    urlcolor=blue,
    pdftitle={Trabalho Prático 1 - Teoria dos Grafos},
}

%--| Personalização dos títulos |--
\titleformat{\section}{\normalfont\bfseries}{\thesection}{1em}{}
\titleformat{\subsection}{\normalfont\bfseries}{\thesubsection}{1em}{}

%----------| Início do documento |----------
\begin{document}

%-----| CAPA |------
\begin{titlepage}
    \centering
    \vspace*{2cm}
    \textbf{\large PONTIFÍCIA UNIVERSIDADE CATÓLICA DE MINAS GERAIS}\\
    
    \vspace*{0.5cm}
    \textbf{\large CIÊNCIA DA COMPUTAÇÃO}\\
    
    \vspace*{4cm}
    \textbf{\large TRABALHO 1}\\
    
    \vspace*{0.5cm}
    \textit{\large Teoria de Grafos e Computabilidade}\\
    
    \vfill
    \textbf{\large Gabriel Cunha Schlegel \\ Kaiky França da Silva \\ Samuel Horta de Faria}\\
    
    \vspace*{1.5cm}
    \textbf{\large Belo Horizonte}\\
    \vspace*{0.5cm}
    \textbf{\large Março de 2024}
\end{titlepage}

%--| SUMÁRIO |--
\tableofcontents
\newpage

%--| INTRODUÇÃO |--
\section{Introdução}

Este trabalho apresenta uma análise comparativa entre dois algoritmos para detecção de ciclos em um grafo: \textbf{Busca em Profundidade} (DFS) e \textbf{Permutação}. Ambos foram implementados utilizando listas de adjacência \textbf{(vector<vector<int>>)} para representar o grafo, já que possui uma maior eficiência em termos de tempo de acesso e uso de memória, comparada a representação porr meio de matriz de adjacência.  

\newpage

%--| METODOLOGIA |--
\section{Metodologia}

\subsection{Busca em Profundidade (DFS)}
O algoritmo de DFS percorre o grafo explorando caminhos e armazenando os vértices visitados. Caso retorne a um vértice já visitado e o ciclo tenha pelo menos três nós, ele é identificado. Para evitar contagem duplicada, foi utilizado um conjunto \textbf{(set<vector<int>>)} para armazenar apenas ciclos únicos.  

\subsection{Permutação}
O método de permutação gera todas as possíveis sequências de vértices e verifica quais delas formam ciclos válidos. Embora garanta a identificação de todos os ciclos, esse método possui um custo computacional \textbf{extremamente alto}.

\newpage

%--| ANÁLISE COMPARATIVA |--
\section{Análise Comparativa}

A principal diferença entre os métodos está no desempenho conforme o tamanho do grafo aumenta:

\begin{itemize}
    \item \textbf{DFS}: mais rápido e eficiente para grafos pequenos, mas pode consumir muita memória devido às chamadas recursivas em grafos grandes.
    \item \textbf{Permutação}: extremamente preciso, porém inviável para grafos grandes, pois seu tempo de execução cresce de forma significativa.
\end{itemize}

\newpage

%--| CONCLUSÃO |--
\section{Conclusão}

O algoritmo de \textbf{DFS} demonstrou ser a melhor opção para detecção de ciclos, pois mantém um bom equilíbrio entre eficiência e uso de memória. O método de permutação, apesar de garantir a exatidão na identificação de ciclos, é inviável para grafos grandes devido ao seu crescimento fatorial.

A escolha da lista de adjacência como estrutura de representação do grafo proporcionou um melhor desempenho, evitando custos excessivos de armazenamento.

\end{document}
